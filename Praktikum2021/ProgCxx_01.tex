\documentclass[a4paper,11pt,bahasa]{article} % screen setting

\usepackage[a4paper]{geometry}
\geometry{verbose,tmargin=1.5cm,bmargin=1.5cm,lmargin=1.5cm,rmargin=1.5cm}

\setlength{\parskip}{\smallskipamount}
\setlength{\parindent}{0pt}

%\usepackage{fontspec}
\usepackage[libertine]{newtxmath}
\usepackage[no-math]{fontspec}
\setmainfont{Linux Libertine O}
\setmonofont{DejaVu Sans Mono}

\usepackage{hyperref}
\usepackage{url}
\usepackage{xcolor}

\usepackage{graphicx}
\usepackage{float}

\usepackage{minted}

\newminted{cpp}{breaklines,fontsize=\footnotesize}
\newminted{bash}{breaklines,fontsize=\footnotesize}
\newminted{text}{breaklines,fontsize=\footnotesize}

\newcommand{\txtinline}[1]{\mintinline[breaklines,fontsize=\footnotesize]{text}{#1}}
\newcommand{\cppinline}[1]{\mintinline[breaklines,fontsize=\footnotesize]{cpp}{#1}}

\definecolor{mintedbg}{rgb}{0.90,0.90,0.90}
\usepackage{mdframed}
\BeforeBeginEnvironment{minted}{
    \begin{mdframed}[backgroundcolor=mintedbg,%
        topline=false,bottomline=false,%
        leftline=false,rightline=false]
}
\AfterEndEnvironment{minted}{\end{mdframed}}


%\usepackage{setspace}
%\onehalfspacing

\usepackage{appendix}


\newcommand{\highlighteq}[1]{\colorbox{blue!25}{$\displaystyle#1$}}
\newcommand{\highlight}[1]{\colorbox{red!25}{#1}}


\newcounter{soal}%[section]
\newenvironment{soal}[1][]{\refstepcounter{soal}\par\medskip
   \noindent \textbf{Soal~\thesoal. #1} \sffamily}{\medskip}


\definecolor{mintedbg}{rgb}{0.95,0.95,0.95}
\BeforeBeginEnvironment{minted}{
    \begin{mdframed}[backgroundcolor=mintedbg,%
        topline=false,bottomline=false,%
        leftline=false,rightline=false]
}
\AfterEndEnvironment{minted}{\end{mdframed}}


\BeforeBeginEnvironment{soal}{
    \begin{mdframed}[%
        topline=true,bottomline=true,%
        leftline=true,rightline=true]
}
\AfterEndEnvironment{soal}{\end{mdframed}}


\begin{document}


\title{Pemrograman C++}
\author{Fadjar Fathurrahman}
\date{}
\maketitle

\section{Tujuan}

\begin{itemize}
\item Membuat program C++ sederhana
\end{itemize}



\section{Teori Singkat}


\subsection{Struktur Dasar Program C++}

Suatu program C++ sederhana dapat berbentuk sebagai berikut.
\begin{cppcode}
// Program Hello
// Menampilkan teks ke layar terminal
// 
// Author: Dudun
// (c) 2021

#include <iostream>

using namespace std;

int main()
{
  // Kode program
  cout << "Halo, nama saya adalah Dudun" << endl;

  return 0;
}
\end{cppcode}

Kode C++ berbentuk file teks dengan ekstensi \txtinline{.cpp},
\txtinline{.cxx} atau \txtinline{.C}

Penjelasan program:

\begin{cppcode}
// Program Hello
// Menampilkan teks ke layar terminal
// 
// Author: Dudun
// (c) 2021
\end{cppcode}
Baris yang diawali dengan tanda \txtinline{//} menyatakan komentar, yaitu bagian
program yang ditujukan untuk manusia dan komputer akan mengabaikan teks yang ada
pada komentar. Komentar pada bagian awal biasanya berisi:
\begin{itemize}
\item penjelasan mengenai kode yang dibuat, misalnya apa yang akan dilakukan oleh
    program,
\item nama programmer,
\item lisensi program, dan
\item info lainnya
\end{itemize}

\begin{cppcode}
#include <iostream>
\end{cppcode}

Bagian yang diawali dengan tanda \# menyatakan direktif.
Dalam program ini kita memberitahu compiler bahwa kita akan menggunakan
pustaka standard untuk input-output dengan menyertakan header bernama
\txtinline{iostream}.


\begin{cppcode}
using namespace std;
\end{cppcode}




\begin{cppcode}
int main()
{
  // Kode program
  cout << "Halo, nama saya adalah Dudun" << endl;

  return 0;
}
\end{cppcode}

Titik awal eksekusi dari dari suatu program C++ adalah sebuah fungsi
bernama \cppinline{main}.

Fungsi ini memiliki tipe kembalian
\cppinline{int}. Kode program yang membangun fungsi ini diketikkan
dalam tanda kurung kurawal \cppinline{ {} }



Alternatif:
\begin{cppcode}
#include <iostream>

int main()
{
  std::cout << "Halo, nama saya adalah Dudun" << std::endl;
  return 0;
}
\end{cppcode}


\subsection{Input-Output Sederhana}

\begin{cppcode}
cout << 
\end{cppcode}

Operator \cppinline{<<} dikenal dengan insertion operator.

Operator \cppinline{>>} dikenal dengan extraction operator.

\begin{cppcode}
cin >>
\end{cppcode}

Penggunaan getline


\section{Praktikum}

Test



\section{Pekerjaan Rumah}

\begin{soal}[Sejarah C++]
Sumber internet sejarah singkat C++.
Siapa yang mendesain, latar belakang mengapa dikembangkan C++,
\end{soal}


\begin{soal}
\end{soal}


\bibliographystyle{unsrt}
\bibliography{BIBLIO}

\end{document}
